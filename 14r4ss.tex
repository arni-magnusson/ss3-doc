
\section{Using R To View Model Output (r4ss)}\label{r4ss}

A collection of functions developed as a package, \texttt{r4ss}, for the statistical software R has been created to explore SS model output.  The functions include tools for summarizing and plotting results, manipulating files, visualizing model parameterizations, and various other tasks.  Currently, information on the code, including installation instructions, can be found on \href{https://github.com/r4ss/r4ss}{Github}.  The software package is under constant development to maintain compatibility with new versions of SS and to improve functionality.

Two of the most commonly used functions for model diagnostics are \texttt{SS\_output} and \texttt{SS\_plots}.  After running a model using SS, the report can be read into R by the \texttt{SS\_output} function which stores quantities in a list with named objects.  This list can then be passed to the \texttt{SS\_plots} function which creates a series of over 100 plots that are useful to visualize output such as model fit to the data and time series of quantities of interest.

The latest version of \texttt{r4ss} can be installed directly from GitHub at any time via the \texttt{remotes} package in R with the following commands:

\begin{quote}
	\begin{verbatim}
	> install.packages("remotes")
	> remotes::install_github("r4ss/r4ss")
	\end{verbatim}
\end{quote}

%This installs the main branch of the package from \href{https://github.com/r4ss/r4ss}{Github} which is a well tested version that should work across multiple versions of SS.  

%The package undergoes continuous development in a separate location where new techniques for visualizing data and outputs along with code changes/corrections to align with new impending releases versions of SS. The development branch can be installed using the following R commands:

%\begin{quote}
%	\begin{verbatim}
%	> install.packages("remotes")
%	> remotes::install_github("r4ss/r4ss", ref = "development")
%	\end{verbatim}
%\end{quote}

Once you have installed the \texttt{r4ss} package, it can be loaded in the regular manner:

\begin{quote}
	\begin{verbatim}
	> library(r4ss)
	\end{verbatim}
\end{quote}

Two of the most commonly used functions for model diagnostics are \texttt{SS\_output} and \texttt{SS\_plots}. After running a model using SS, the output files including Report.sso can be read into R by the \texttt{SS\_output} function which stores quantities in a list with named objects.  This list can then be passed to the \texttt{SS\_plots} function, which creates a series of over 100 plots that are useful to visualize output such as model fits to the data. Here is an example call:

\begin{quote}
	\begin{verbatim}
	> setwd("C:\directory where model was run")
	> base.model = SS_output(getwd())
	> SS_plots(base.model)
	\end{verbatim}
\end{quote}
  
\pagebreak
The functions included in r4ss range from general use to functions developed for specific model applications:
\begin{center}
	\begin{longtable}{p{4.5cm} p{10.52cm}}
		\hline
		Core Functions & \Tstrut\Bstrut\\
		\hline
		SS\_output \Tstrut& A function to create a list object for the output from Stock Synthesis\\
		SS\_plots  \Tstrut& Plot many quantities related to output from Stock Synthesis\\
		\hline
		
		\multicolumn{2}{l}{Individual plotting functions called by SS\_plots:} \Tstrut\Bstrut\\
		\hline
		SSplotBiology \Tstrut& Plot biology related quantities from Stock Synthesis model output, including mean weight, maturity, fecundity, and spawning output. \\
		SSplotCatch   \Tstrut & Plot catch related quantities \\
		SSplotCohorts \Tstrut & Plot cumulative catch by cohort \\
		SSplotComps   \Tstrut & Plot composition data and fits \\
		SSplotData    \Tstrut & Timeline of presence/absence data by type, year, and fleet \\
		SSplotDiscard \Tstrut & Plot fit to discard fraction \\
		SSplotIndices \Tstrut & Plot indices of abundance and associated quantities \\
		SSplotMnwt    \Tstrut & Plot mean weight data and fits \\
		SSplotMovementMap \Tstrut & Show movement rates on a map \\
		SSplotMovementRates \Tstrut & Show movement rates on a map \\
		SSplotNumbers \Tstrut& Plot numbers-at-age related data and fits \\
		SSplotRecdevs \Tstrut& Plot recruitment deviations \\
		SSplotRecdist \Tstrut& Plot of recruitment distribution among areas and seasons \\
		SSplotSelex   \Tstrut& Plot selectivity \\
		SSplotSexRatio \Tstrut& Plot sex ratios \\
		SSplotSummaryF \Tstrut& Plot time series summary of F (or harvest rate) \\
		SSplotSpawnrecruit \Tstrut& Plot spawner-recruit curve \\
		SSplotSPR     \Tstrut& Plot SPR quantities \\
		SSplotTags    \Tstrut& Plot tagging data and fits \\
		SSplotTimeseries \Tstrut& Plot time series data \\
		SSplotYield   \Tstrut& Plot yield and surplus production \\
		SS\_html \Tstrut& Create HTML files to view figures in browser \\
		\hline
	
		\multicolumn{2}{l}{Model comparisons and other diagnostics:} \Tstrut\Bstrut\\
		\hline
		SSsummarize   \Tstrut & Function to read output from multiple SS models\\
		SStableComparison \Tstrut & Make table comparing quantities across models\\
		SSplotComparison \Tstrut & Plot output from multiple SS models \\
		SSplotPars    \Tstrut & Plot distributions of priors, posteriors, and estimates \\
		SS\_profile \Tstrut & Run likelihood parameter profiles \\
		SSplotProfile \Tstrut & Plot likelihood profile results \\
		PinerPlot     \Tstrut & Plot fleet-specific contributions to likelihood profile \\
		SS\_RunJitter \Tstrut & Run multiple model jitters to determine best model fit \\
		SS\_doRetro \Tstrut & Run retrospective analysis \\
		SSmohnsrho \Tstrut & Calculate Mohn's Rho values\\
		SSplotRetroRecruits \Tstrut & Make retrospective pattern of recruitment estimates (a.k.a. squid plot) as seen in Pacific hake assessments\Bstrut \\
		SS\_fitbiasramp \Tstrut& Estimate bias adjustment for recruitment deviates \Bstrut\\
		\hline
		
		\multicolumn{2}{l}{File manipulation for inputs:}\Tstrut\Bstrut\\
		\hline
		SS\_readdat  \Tstrut & Read data file \\
		SS\_readforecast \Tstrut & Read forecast file \\
		SS\_readstarter  \Tstrut & Read starter file \\
		SS\_writedat  \Tstrut & Write data file \\
		SS\_writeforecast \Tstrut & Write forecast file \\
		SS\_writestarter  \Tstrut & Write starter file \\\Tstrut 
		SS\_makedatlist   \Tstrut & Make a list for SS data \\
		SS\_parlines    \Tstrut & Get parameter lines from SS control file \\
		SS\_changepars  \Tstrut  & Change parameters in the control file \\
		SSmakeMmatrix    \Tstrut & Create inputs for entering a matrix of natural mortality by age and year \\
		SS\_profile      \Tstrut & Run a likelihood profile in SS (incomplete) \\
		NegLogInt\_Fn     \Tstrut& Calculated variances of time-varying parameters using SS implementation of the Laplace Approximation \Bstrut\\
		\hline
		
		\multicolumn{2}{l}{File manipulations for outputs:}\Tstrut\Bstrut\\
		\hline
		SS\_recdevs  \Tstrut & Insert a vector of recruitment deviations into the control file \\
		SS\_splitdat \Tstrut & Split apart bootstrap data to make input file \Bstrut\\
		\hline
		
		\multicolumn{2}{l}{Functions related to MCMC diagnostics:}\Tstrut\Bstrut\\
		\hline
		SSgetMCMC      \Tstrut & Read MCMC output \\
		SSplotMCMC\_ExtraSelex \Tstrut & Plot uncertainty around chosen selectivity ogive from MCMC \\
		mcmc.nuisance \Tstrut & Summarize nuisance MCMC output \\
		mcmc.out      \Tstrut & Summarize, analyze, and plot key MCMC output \Bstrut\\
		\hline
				
%		\multicolumn{2}{l}{Interactive tools for exploring functional forms:} \Tstrut\Bstrut\\
%		\hline
%		movepars    \Tstrut  & Explore movement parameterization \\
%		selfit      \Tstrut  & A function to visualize parameterization of double normal and double logistic selectivity in SS \\
%		selfit\_spline \Tstrut & Visualize parameterization of cubic spline selectivity in SS \\
%		sel\_line   \Tstrut & A function for drawing selectivity curves\Bstrut \\
%		\hline
%
%		\multicolumn{2}{l}{Minor plotting functions:}\Tstrut\Bstrut\\
%		\hline
%		bubble3         \Tstrut & Create a bubble plot \\
%		make\_multifig  \Tstrut & Create multi-figure plots \\
%		Make\_multifig\_sexratio \Tstrut & Create multi-figure plots of sex ratios \\
%		plotCI          \Tstrut & Plot points with confidence intervals \\
%		rich\_colors\_short \Tstrut & Make a vector of colors \\
%		stackpoly       \Tstrut & Plot stacked polygons \\
%		mountains       \Tstrut & Make shaded polygons with a mountain-like appearance \Bstrut\\
		
%		\hline
%		\multicolumn{2}{l}{Really specialized functions:} \Tstrut\Bstrut\\
%		\hline
%		DoProjectPlots \Tstrut & Make plots from Rebuilder program \\
%		IOTCmove       \Tstrut & Make a map of movement for a 5-area Indian Ocean model \\
%		SSFishGraph    \Tstrut & A function for converting SS output to the format used by FishGraph \\
%		TSCplot        \Tstrut & Create a plot for the TSC report \Bstrut\\
%		\hline
	\end{longtable}
\end{center}

\pagebreak
