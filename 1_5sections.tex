
 \section{Introduction}\label{sec:intro}
	This manual provides a guide for using the stock assessment program, Stock Synthesis (SS).  The guide contains a description of the input and output files and usage instructions. A technical description of the model itself is in Methot and Wetzel (2013).  SS is programmed using Auto Differentiation Model Builder (ADMB; Fournier 2001.  ADMB is now available at admb-project.org).  SS currently is compiled using ADMB version 11.1 using the Microsoft C++ Optimizing Compiler Version 16.0.  The model and a graphical user interface are available from the NOAA Fisheries Stock Assessment Toolbox website: nft.nefsc.noaa.gov.  An output processor package, r4ss, in R is available for download from CRAN or GitHub.   Additional information about the package can be located at github.com/r4ss/r4ss .
	
\section{New Features Available in Version 3.30}
		Stock Synthesis version 3.30 has a number of new features. Additional information on each new feature available by clicking on the "Item".
		
		\begin{center}
			\begin{longtable}{p{2cm} p{3cm} p{10cm}}
				Category & Item & Description\\
				\hline
				\endfirsthead
		
				Category & Item & Description\\
				\hline
				\endhead
		
				\hline
				\endfoot
		
				\endlastfoot
				
				General & 
					\hyperlink{GenericFleets}{Generic Fleets} & 
						Fleet specification section of data file is much changed and now includes fleet type, so fishery fleets, bycatch fleets, surveys, and someday predators are specified in any order\\
				  \\
				  & \hyperlink{ListBased}{List-oriented inputs} & 
					    Rather than specify the number of items to be read, now SS can figure it out on its own with lists terminated by -9999 in first field of the read vector \\
				  \\					  
				  & \hyperlink{SubSeas}{Internal sub-seasons} & 
					    SS v3.24 inherently has 2 subseasons each season (begin and middle) at which the age-length-key (ALK) is calculated; now user specifies an even number of sub-seasons to use (2 to ~10) \\
				  \\
				  & \hyperlink{ObsTiming}{Observation Timing} & 
					    Timing of observations now is input as year, month where month is real; e.g. April 15 is 4.5; age-length-key (ALK) used for each observation is calculated to the nearest sub-season.  Old "survey\_timing" replaced by the month specific inputs.  Season is calculated at runtime from the input month and the input season durations. \\
				  \\
				  & \hyperlink{ALK}{Speed} & 
					    Smarter at when to re-calculate the age-length-key (ALK); trims tails of size-at-age so calculations avoid many inconsequential cells of the age-length matrix. ALK tail compression is specified in ths starter file.\\
				  \\				
				  & \hyperlink{Convert} {Converter} & 
					    Special version of SS, SS\_trans, will read files in 3.24 format and write .ss\_new files in 3.30 format.  This is the advised method for converting previous version files.\\
				  \\
				  & \hyperlink{WAA} {Empirical Weight-at-Age} & Implementing empirical weight-at-age is now specified specifically in the control file rather than under the maturity options.\\
				\hline
				Fishery and Catch & 
					\hyperlink{CatchMult}{Catch multiplier} & 
						Each fishing fleet's catch can now have a "q" that is a parameter in the Mgparm section.\\
				  \\						
					& \hyperlink{CatchFormat}{Catch input} & 
						Catch input now as list:  yr, seas, fleet, amount, se. \\
				  \\						
					& \hyperlink{CompTiming}{Observations} & 
						Fishery composition observations can be related to season long catch-at-age, or to a month-specific timing.\\
				  \\					
					& \hyperlink{DomeRetention}{Retention} & 
						Option for dome-shaped retention function. \\
				\hline
				Survey  
					& \hyperlink{Qsetup}{Link functions} & 
						Q\_power is now one of several, and growing, set of link functions. \\
				  \\						
					& \hyperlink{Qsetup}{Catchability setup reorganization} & 
						Major reorganization of catchability (Q) setup, including the link specification. \\
				  \\					
					& \multicolumn{1}{l}{\hyperlink{Qsetup}{Q as a parameter}} & 
						Each survey now must have a Q parameter and its value still can float (as old option 5).\\
				\hline
				Recruitment
					& \hyperlink{SRR}{Stock-Recruitment Parameter Section} & Setup for the stock-recruitment relationship (SRR) parameter section has been changed to allow for blocking around unfished recruitment (R0) and steepness.\\ 
					\\
					& \hyperlink{Shepard}{Shepard SRR} & 
						A 3-parameter Shepard stock-recruitment curve is now an option.\\
					\\
					& \hyperlink{RecrTiming}{Recruitment timing} & 
						Replace "birthseason" with "settlement event" that has explicit timing offset from spawning.  Month of spawning and each settlement event must be specified and need not be at beginning of a season.\\
				\hline
				Benchmark 
					& Global MSY &  
						Global MSY based on knife edge age selection; also do calculation with single age selection. The global MSY value will automatically be included in the report file.\\
				  \\					
					& Mean recruitment distribution & 
						In multi-area model, can now specify range of years to use for the average recruitment distribution for forecasting. This feature is not yet implemented. \\
				  \\
				\hline
				Forecast & 
					Process error & 
						Propagate random walk in mgparms, catchability, and selectivity into forecast. Specifying the end year for process error in the forecast period will implement this option.  This option has only been partial implemented at this junction and will be completed in later versions.\\
				\hline
				Biology 
					& \hyperlink{MGorder}{Parameter order} & 
						Mgparms now have maturity, fecundity, sex ratio, and weight-length by growth pattern.\\
				  \\						
				    & \hyperlink{SexRatio}{Sex ratio} & 
					    Change sex ratio at birth from a constant to a morph-specific MGparameter. \\
				\hline
				Statistical 
					& \hyperlink{GcompVar}{Input variance adjuster} & 
						Added variance adjustment factor for generalized size comp. \\
				  \\						
					& Deviation vectors & 
						Variance of deviation vectors is now specified with 2 parameters for standard error and auto-correlation (rho).\\
				  \\						
					& \hyperlink{Dirichlet}{Dirichlet multinomial} & 
						Dirichlet multinomial now a fleet-specific option; takes one parameter per. \\
				\hline
				Parameters 
					& Density dependence & 
						Beginning of year summary biomass and the rec\_dev parameter are mapped to the    "environmental" matrix so that parameters can be density-dependent.\\
				  \\						
					& Re-order & 
						Pay attention to the new order of the time-varying adjustments to parameters. \\
				  \\						
					& Time-varying parameters & 
						Long parameter lines for spawner-recruit relationship (SRR), catchability (Q), and tag parameters and complete re-vamp of the way that time-varying parameters are implemented for SRR and Q.  Now shares same internal code as mortality-growth and selectivity parameters for time-varying capabilities.\\
				\hline	
			\end{longtable}
		\end{center}

		
\section{File Organization}\label{FileOrganization}		
	\subsection{Input Files}
	\begin{enumerate}
		\item starter.ss:   required file containing filenames of the data file and the control file plus other run controls (required).
		\item datafile:  file containing model dimensions and the data with file extension .dat (required)
		\item control file:  file containing set-up for the parameters with file extension .ctl (required)
		\item forecast.ss:  file containing specifications for forecasts (required) 
		\item ss.par:  previously created parameter file that can be read to overwrite the initial parameter values in the control file (optional)
		\item runnumber.ss:  file containing a single number used as runnumber in output to CumReport.sso and in the processing of profilevalues.ss (optional)
		\item profilevalues.ss:  file contain special conditions for batch file processing (optional)
	\end{enumerate}
	
	\subsection{Output Files}
	\begin{enumerate}
		\item ss.par, ss.std, ss.rep, ss.cor etc.  standard ADMB output files
		\item echoinput.sso:  This file is produced while reading the input files and includes an annotated echo of the input.  The sole purpose of this output file is debugging input errors.
		\item warning.sso:  This file contains a list of warnings generated during program execution.
		\item checkup.sso:  Contains details of selectivity parameters and resulting vectors.  This is written during the first call of the objective function.
		\item Report.sso:  This file is the primary report file.
		\item CompReport.sso:  Beginning with version 3.03, the composition data has been separated into a dedicated report
		\item Forecast-report.sso:  Output of management quantities and for forecasts
		\item CumReport.sso:  This file contains a brief version of the run output, output is appended to current content of file so results of several runs can be collected together.  This is useful when a batch of runs is being processed.
		\item Covar.sso:  This file replaces the standard ADMB ss.cor with an output of the parameter and derived quantity correlations in database format
		\item data.ss\textunderscore new:  contains a user-specified number of datafiles, generated through a parametric bootstrap procedure, and written sequentially to this file
		\item control.ss\textunderscore new:  Updated version of the control file with final parameter values replacing the Init parameter values.
		\item starter.ss\textunderscore new:  New version of the starter file with annotations
		\item Forecast.ss\textunderscore new:  New version of the forecast file with annotations.
		\item rebuild.dat:  Output formatted for direct input to Andre Punt's rebuilding analysis package.  Cumulative output is output to REBUILD.SS (useful when doing MCMC or profiles).
	\end{enumerate}

	
	\subsection{Auxiliary Files}
	\begin{enumerate}
		\item SSv330-OUTPUT.XLS:   Excel file with macros to read report.sso and display results
		\item SELEX24\textunderscore dbl\textunderscore normal.XLS:
		\begin{enumerate}
			\item This excel file is used to show the shape of a double normal selectivity (option number 20 for age-based and 24 for length-based selectivity) given user-selected parameter values.
			\item Instructions are noted in the XLS file but, to summarize
			\begin{enumerate}
				\item Users should only change entries in a yellow box. 
				\item Parameter values are changed manually or using sliders, depending on the value of cell I5.
			\end{enumerate}
			\item It is recommend that users select plausible starting values for double-normal selectivity options, especially when estimating all 6 parameters
			\item Please note that the XLS does NOT show the impact of setting parameters 5 or 6 to ''-999''.  In SS v3.30, this allows the the value of selectivity at the initial and final age or length to be determined by the shape of the double-normal arising from parameters 1-4, rather than forcing the selectivity at the intial and final age or length to be estimated separately using the value of parameters 5 and 6. 
		\end{enumerate}
		\item SELEX17\textunderscore age\textunderscore randwalk.XLS:
		\begin{enumerate}
			\item This excel file is used to show the shape of age-based selectivity arising from option 17 given user-selected parameter values
			\item Users should only change entries in the yellow box.
			\item The red box is the maximum cumulative value, which is subtracted from all cumulative values.  This is then exponentiated to yield the estimated selectivity curve.  Positive values yield increasing selectivity and negative values yield decreasing selectivity.
		\end{enumerate}
		\item PRIOR-TESTER.XLS:
		\begin{enumerate}
			\item The 'compare' tab of this spreadsheet shows how the various options for defining parameter priors work
		\end{enumerate}
		\item SS-Control\textunderscore Setup.XLS:
		\begin{enumerate}
			\item Shows how to setup an example control file for SS
		\end{enumerate}
		\item SS-Data\textunderscore Input.XLS:
		\begin{enumerate}
			\item Shows how to setup an example data input for SS
		\end{enumerate}
		\item Growth.XLS: 
		\begin{enumerate}
			\item Excel file to test parameterization between the growth curve options within SS.
			\item Instructions are noted in the XLS file but, to summarize
			\begin{enumerate}
				\item Users should only change entries in a yellow box.  
				\item Entries in a red box are used internally, and can be compared with other parameterizations, but should not be changed.
			\end{enumerate}
			\item The SS-VB is identical to the standard VB, but uses a parameterization where length is estimated at pre-defined ages, rather than A=0 and A=Inf.  The Schnute- Richards is identical to the Richards-Maunder, but similarly uses the parameterization with length at pre-defined ages.  The Richards coefficient controls curvature, and if the curvature coefficient = 1, it reverts to the standard VB curve. 
		\end{enumerate}
		\item Movement.XLS:
		\begin{enumerate}
			\item Excel file to explore SS movement parameterization
		\end{enumerate}
	\end{enumerate}
		
\section{Starting SS}
SS is typically run through the command line interface although it can also be called from another program such as R or the SS-GUI or a script file (such as a DOS batch file). SS is compiled for Windows, Mac, and Linux operating systems. The memory requirements depend on the complexity of the model you run, but in general, SS will run much slower on computers with inadequate memory. See the section \ref{RunningSS} for additional notes on methods of running SS.

Communication with the program is through text files.  When the program first starts, it reads the file starter.ss, which typically must be located in the same directory from which SS is being run.  The file starter.ss contains required input information plus references to other required input files, as described in section \ref{FileOrganization}.  The names of the control and data files must match the names specified in the starter.ss file.  File names, including starter.ss, are case-sensitive on Linux and Mac systems but not on Windows. Output from SS is as text files containing specific keywords.  Output processing programs, such as the SS GUI, Excel, or R can search for these keywords and parse the specific information located below that keyword in the text file.



\section{Computer Requirements and Recommendations}
SS is compiled to run under DOS with a 32-bit or 64-bit Windows, Linux, and Apple operating systems.  It is recommended that the computer have at least a 2.0 Ghz processor and 2 GB of RAM. 