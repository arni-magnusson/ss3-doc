\begin{landscape}

\section{Appendix E: Example Model Files}


\subsection{starter.ss}

\begin{verbatim}
	#V3.30a
	#C starter comment here
	simple_disc.dat
	simple_lendisc.ctl
	0 # 0=use init values in control file; 1=use ss3.par
	1 # run display detail (0,1,2)
	1 # detailed age-structured reports in REPORT.SSO (0,1) 
	0 # write detailed checkup.sso file (0,1) 
	1 # write to cumreport.sso (0=no,1=like&timeseries; 2=add survey fits)
	1 # Include prior_like for non-estimated parameters (0,1) 
	1 # Use Soft Boundaries to aid convergence (0,1) (recommended)
	3 # Number of datafiles to produce: 1st is input, 2nd is estimates, 3rd and higher are bootstrap
	100 # Turn off estimation for parameters entering after this phase
	0 # MCeval burn interval
	1 # MCeval thin interval
	0 # jitter initial parm value by this fraction
	1969 # min yr for sdreport outputs (-1 for styr)
	2004 # max yr for sdreport outputs (-1 for endyr; -2 for endyr+Nforecastyrs
	0 # N individual STD years 
	#vector of year values 	
	0.0001 # final convergence criteria (e.g. 1.0e-04) 
	0 # retrospective year relative to end year (e.g. -4)
	1 # min age for calc of summary biomass
	1 # Depletion basis:  denom is: 0=skip; 1=rel X*B0; 2=rel X*Bmsy; 3=rel X*B_styr
	0.4 # Fraction (X) for Depletion denominator (e.g. 0.4)
	1 # SPR_report_basis:  0=skip; 1=(1-SPR)/(1-SPR_tgt); 2=(1-SPR)/(1-SPR_MSY); 3=(1-SPR)/(1-SPR_Btarget);
	# 4=rawSPR
	4 # F_report_units: 0=skip; 1=exploitation(Bio); 2=exploitation(Num); 3=sum(Frates); 4=true F for range 
	# of ages
	20 23 #_min and max age over which average F will be calculated
	1 # F_std_basis: 0=raw_F_report; 1=F/Fspr; 2=F/Fmsy ; 3=F/Fbtgt
	0 # ALK tolerance (example 0.0001)
	3.3 # check value for end of file and for version control
\end{verbatim}

\subsection{forecast.ss}

\begin{verbatim}
	#V3.30a
	1 # Benchmarks: 0=skip; 1=calc F_spr,F_btgt,F_msy 
	2 # MSY: 1= set to F(SPR); 2=calc F(MSY); 3=set to F(Btgt); 4=set to F(endyr) 
	0.4 # SPR target (e.g. 0.40)
	0.342 # Biomass target (e.g. 0.40)
	#_Bmark_years: beg_bio, end_bio, beg_selex, end_selex, beg_relF, end_relF (enter actual year, or values of
	# 0 or -integer to be rel. endyr)
	0 0 0 0 0 0
	#  2001 2001 2001 2001 2001 2001 # after processing 
	1 #Bmark_relF_Basis: 1 = use year range; 2 = set relF same as forecast below
	#
	1 # Forecast: 0=none; 1=F(SPR); 2=F(MSY) 3=F(Btgt); 4=Ave F (uses first-last relF yrs); 5=input annual F
	# scalar
	3 # N forecast years 
	0.2 # F scalar (only used for Do_Forecast==5)
	#_Fcast_years:  beg_selex, end_selex, beg_relF, end_relF  (enter actual year, or values of 0 or -integer to
	# be rel. endyr)
	0 0 -10 0
	#  2001 2001 1991 2001 # after processing 
	1 # Control rule method (1=catch=f(SSB) west coast; 2=F=f(SSB) ) 
	0.4 # Control rule Biomass level for constant F (as frac of Bzero, e.g. 0.40); (Must be > the no F level
	# below) 
	0.1 # Control rule Biomass level for no F (as frac of Bzero, e.g. 0.10) 
	0.75 # Control rule target as fraction of Flimit (e.g. 0.75) 
	3 #_N forecast loops (1=OFL only; 2=ABC; 3=get F from forecast ABC catch with allocations applied)
	3 #_First forecast loop with stochastic recruitment
	0 #_Forecast loop control #3 (reserved for future bells&whistles) 
	0 #_Forecast loop control #4 (reserved for future bells&whistles) 
	0 #_Forecast loop control #5 (reserved for future bells&whistles) 
	2010  #FirstYear for caps and allocations (should be after years with fixed inputs) 
	0 # stddev of log(realized catch/target catch) in forecast (set value>0.0 to cause active impl_error)
	0 # Do West Coast gfish rebuilder output (0/1) 
	1999 # Rebuilder:  first year catch could have been set to zero (Ydecl)(-1 to set to 1999)
	2002 # Rebuilder:  year for current age structure (Yinit) (-1 to set to endyear+1)
	1 # fleet relative F:  1=use first-last alloc year; 2=read seas(row) x fleet(col) below
	# Note that fleet allocation is used directly as average F if Do_Forecast=4 
	2 # basis for fcast catch tuning and for fcast catch caps and allocation  (2=deadbio; 3=retainbio; 
	# 5=deadnum; 6=retainnum)
	# Conditional input if relative F choice = 2
	# Fleet relative F:  rows are seasons, columns are fleets
	#_Fleet:  FISHERY1 SURVEY1 SURVEY2
	#  1 0 0
	# enter list of fleet number and max for fleets with max annual catch; terminate with fleet=-9999
	-9999 -1
	# enter list of area ID and max annual catch; terminate with area=-9999
	-9999 -1
	# enter list of fleet number and allocation group assignment, if any; terminate with fleet=-9999
	1 1
	-9999 -1
	#_if N allocation groups >0, list year, allocation fraction for each group 
	# list sequentially because read values fill to end of N forecast
	# terminate with -9999 in year field 
	2002  1
	-9999  1 
	-1 # basis for input Fcast catch: -1=read basis with each obs; 2=dead catch; 3=retained catch; 99=input
	# Hrate(F)
	#enter list of Fcast catches; terminate with line having year=-9999
	#_Year Seas Fleet Catch(or_F) Basis 
	2003 1 1 300 2
	2004 1 1 300 2
	-9999 1 1 0  2 
	#
	999 # verify end of input 
\end{verbatim}

\subsection{data.ss}

\subsection{control.ss}

\end{landscape}