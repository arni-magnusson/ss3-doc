\myparagraph{Specification of Time-Varying Parameters in Long Parameter Lines} 
Selecting time-varying properties for a parameter is specified using element 8 - 14 in the long parameter line setup.  Each element and the options for selection are as described here:

\hypertarget{EnvVar}{}
\begin{itemize}

\item Environmental Link and  Variance (element 8)
	\begin{itemize}
		\item The environmental link and variance (env\_var\&link) matrix is populated with the read environmental data for columns 1-N environmental variables. Alternatively, the environmental link can be linked to derived quantities mapped to columns -1 to -4 where each value corresponds to the following quantities:
		\begin{itemize}
			\item -1;  for ln(relative spawning biomass);
			\item -2;  for recruitment deviation;
			\item -3;  for ln(relative summary biomass) (e.g., current year summary biomass divided by the unfished summary biomass);
			\item -4;  for ln(relative summary numbers).
		\end{itemize}
		\item So, environmental input 103 would use link type 1 (specified in element 9) and apply it to environmental data column 3 and environmental input -103  would use link type 1 and apply it to the "-3" column which is ln(relative summary biomass).
		\item These four derived quantities are all calculated at the beginning of each year within the model, so they are available inside SS to use as the basis for time-varying parameter links without violating any oder of operations rule. 
	\end{itemize}
	
\item Deviation Link (element 9)
	\begin{itemize}
		\item 1 = multiplicative ($P_y*=exp(\text{dev}_y*\text{dev}_{se}$),
		\item 2 = additive ($P_y+=\text{env}_y*\text{dev}_{se}$),
		\item 3 = random walk options are now implemented by using rho in the objective function. SS now expects the estimated deviations to be normal in distribution and the deviation values are multiplied by the standard error parameter as they are used,
		\item 4 = zero reverting random walk with rho. The deviation parameter is now multiplied by the standard error parameter, rather than deviations being penalized according to a specified standard error (the approach in SS v.3.24).
		\item 5 = zero reverting random walk with rho and a logit transformation to stay within the minimum and maximum parameter bounds (approach added in SS v.3.30.16).
		\item The option of applying the final model year deviation value into the forecast period was added in v. 3.30.13.  This new option is specified by selecting the appropriate deviation link option (1, 2, 3, or 4) and appending a 2 at the front (21, 22, 23, or 24) which will use the final year deviation value for all forecast years. 
	\end{itemize}
	
\item Deviation  Minimum Year (element 10)
	\begin{itemize}
		\item Year for deviations to start for parameter
	\end{itemize}
	
\item Deviation  Maximum Year (element 11)
	\begin{itemize}
		\item Year for deviations to end for parameter
	\end{itemize}
	
\item Deviation Phase (element 12)
	\begin{itemize}
		\item integer, this available element in the long parameter line is now a deviation vector specific phase control
	\end{itemize}
	
\item Blocks (element 13)
	\begin{itemize}
		\item >0: time block index for parameter.
		\item -1: trend with final as offset from base parameter and offset values is in natural log space, also inflection year is in natural log space and the offset from ln(0.5). No additional parameter lines are required.  Three parameters will be estimated; end trend parameter value logistic offset, inflection year logistic offset, and slope.
		\item -2: trend with final as standalone value. No additional parameter lines are required. Three parameters will be estimated; end trend parameter value, inflection year, and slope.
		\item -3 end value is a fraction of base parameter maximum - minimum; inflection year is fraction of end year - start year. No additional parameter lines are required. Three parameters will be estimated; end trend parameter value as a fraction, inflection year as a fraction, and slope.
	\end{itemize}
	
\item Block Functional Form (element 14)
	\begin{itemize}
		\item 0: multiplicative parameter ($P_{y} = P_{base}*exp(tv\_para)$),
		\item 1: additive parameter ($P_{y} = P_{base} + tv\_para$),
		\item 2: replace parameter ($P_{y} = tv\_para$),
		\item 3: random walk across blocks ($P_{block} = P_{block,-1} + tv\_para$),
		\item 4: mean reverting random walk
	\end{itemize}
\end{itemize}