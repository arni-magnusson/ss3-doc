% Preamble =========================================================================================
%\input{preamble}

\documentclass[12pt]{article}

\usepackage{lipsum}
\usepackage[margin=1in,left=1in,includefoot]{geometry}
\usepackage{graphicx} %allows for impage import
\graphicspath{{images/}}
\usepackage[hidelinks]{hyper ref} % Allows for clickable links
\usepackage{sectsty}
\usepackage{multirow}
\usepackage{booktabs}
\usepackage[T1]{fontenc}
\usepackage{lmodern}
\usepackage[none]{hyphenat}
\usepackage{array}
\usepackage{lscape}
\usepackage{geometry}
\usepackage{pdflscape}
\usepackage{longtable}
\usepackage[utf8]{inputenc}
\usepackage[english]{babel}
%\usepackage[table]{xcolor}
\usepackage{colortbl}
\usepackage{fixltx2e}
\usepackage{hhline}
\usepackage{dcolumn}
\usepackage{hyperref}
\hypersetup{pdftex,colorlinks=true,allcolors=black}
\usepackage{hypcap}
\usepackage{amsmath}
\usepackage{hyperref}

\definecolor{lightgray}{gray}{0.92}

%\newcolumntype{L}{>{\centering\arraybackslash}m{3in}}
%\newcolumntype{L}{>{\centering\arraybackslash}m{2in}}
%\newcolumntype{L}{>{\centering\arraybackslash}m{1in}}

%\sectionfont{\LARGE}
%\subsectionfont{\Large}
%\subsubsectionfont{\large}

%\usepackage{float}
%\usepackage{fancyhdr}
%\pagestyle{fancy}
%\fancyhead{}

\setlength{\extrarowheight}{1.5pt}
\setlength{\arraycolsep}{1.5pt}


\begin{document}

\begin{center}
	\Large{Stock Synthesis v.3.30\\
	New Available Features\\}
\end{center}

\noindent Stock Synthesis (SS) version v.3.30 was designed specifically to provide more precise temporal control of growth, expected values for data, and for recruitment.  In addition, a large number of new features make substantial changes to the input formats.  Two versions of SS have been developed to assist with converting v.3.24 models to the new expected format for v.3.30.  The transition executable, ss\_trans.exe, reads SS v.3.24 input files and produces SS v.3.30 formatted versions of those input files.  This program can convert nearly every feature in v.3.24.  However, some feature will need to be converted by hand.  The SS executable, ss.exe, will then be your primary new assessment tool.  

\begin{center}
	\begin{longtable}{p{2cm} p{3cm} p{10cm}}
	Category & Item & Description\\
	\hline
	\endfirsthead
	
	Category & Item & Description\\
	\hline
	\endhead
	
	\hline
	\endfoot
	
	\endlastfoot
	
	General & 
	\hyperlink{GenericFleets}{Generic Fleets} & 
	Fleet specification section of data file is much changed and now includes fleet type, so fishery fleets, bycatch fleets, surveys, and someday predators are specified in any order\\
	\\
	& \hyperlink{ListBased}{List-oriented inputs} & 
	Rather than specify the number of items to be read, now SS can figure it out on its own with lists terminated by -9999 in first field of the read vector \\
	\\					  
	& \hyperlink{SubSeas}{Internal sub-seasons} & 
	SS v3.24 inherently has 2 subseasons each season (begin and middle) at which the age-length-key (ALK) is calculated; now user specifies an even number of sub-seasons to use (2 to many) \\
	\\
	& \hyperlink{ObsTiming}{Observation Timing} & 
	Timing of observations now is input as year month where month is real; e.g. April 15 is 4.5; age-length-key (ALK) used for each observation is calculated to the nearest sub-season.  Old "survey\_timing" replaced by the month specific inputs.  Season is calculated at runtime from the input month and the input season durations. \\
	\\
	& \hyperlink{ALK}{Speed} & 
	Smarter at when to re-calculate the age-length-key (ALK); trims tails of size-at-age so calculations avoid many inconsequential cells of the age-length matrix. ALK tail compression is specified in the starter file.\\
	\\				
	& \hyperlink{Convert} {Converter} & 
	Special version of SS, ss\_trans.exe, will read files in 3.24 format and write *.ss\_new files in 3.30 format.  This is the advised method for converting previous version files, but always do a side-by-side comparison.\\
	\\
	& \hyperlink{WAAparm} {Empirical Weight-at-Age} & Implementing empirical weight-at-age is now specified separately in the control file rather than under the maturity options.\\
	\\
	& \hyperlink{Priors}{Prior Type} & Change in the prior numbering for parameters.  Now, 0 indicates no prior, and 6 indicates a normal distribution prior. Options 1-5 remain unchanged.\\
	\hline
	Fishery and Catch & 
	\hyperlink{CatchMult}{Catch multiplier} & 
	Each fishing fleet's catch can now have a "q" that is a parameter in the MGparm section.\\
	\\						
	& \hyperlink{CatchFormat}{Catch input} & 
	Catch is required to be input now as a list:  yr, seas, fleet, amount, se. \\
	\\						
	& \hyperlink{CompTiming}{Observations} & 
	Fishery composition observations can be related to season long catch-at-age, or to a month-specific timing.\\
	\\					
	& \hyperlink{DomeRetention}{Retention} & 
	Option for dome-shaped retention function and for age-based retention. \\
	\hline
	Selectivity 
	& Scaling Options& 	
	A new non-parametric selectivity types that are scaled by the raw values at particular ages, rather than the max age.\\
	\hline
	Survey
	& Special Survey Types & 
	Special selectivity options (type 30 or $>$) are no longer specified within the control file.  Specifying the use of one of these selectivity types is now done within the data file by selecting the survey "units". \\  
	\\
	& \hyperlink{Qsetup}{Link functions} & 
	Q\_power is now one of several, and growing, set of link functions. \\
	\\						
	& \hyperlink{Qsetup}{Catchability setup reorganization} & 
	Major reorganization of catchability (Q) setup, including the link specification. \\
	\\					
	& \multicolumn{1}{l}{\hyperlink{Qsetup}{Q as a parameter}} & 
	Each survey now must have a Q parameter and its value still can float (as old option 5).\\
	\hline
	Recruitment
	& \hyperlink{Shepard}{Shepard SRR} & 
	A 3-parameter Shepard stock-recruitment curve is now an option.\\
	\\
	& \hyperlink{RecrTiming}{Recruitment timing} & 
	Replace "birthseason" with "settlement event" that has explicit timing offset from spawning.  Month of spawning and each settlement event must be specified and need not be at beginning of a season.\\
	\hline
	Benchmark 
	& Global MSY &  
	Global MSY based on knife edge age selection; also do calculation with single age selection. The global MSY value will automatically be included in the report file.\\
	\\					
	& Mean recruitment distribution & 
	In multi-area model, can now specify range of years to use for the average recruitment distribution for forecasting. This feature is not yet implemented. \\
	\hline
	Forecast & 
	Process error & 
	Propagate random walk in MGparms, catchability, and selectivity into forecast. Specifying the end year for process error in the forecast period will implement this option.  This option has only been partial implemented at this junction and will be completed in later versions.\\
	\hline
	Biology 
	& \hyperlink{MGorder}{Parameter order} & 
	MGparms now have maturity, fecundity, sex ratio, and weight-length by growth pattern.\\
	\\						
	& \hyperlink{SexRatio}{Sex ratio} & 
	Change sex ratio at birth from a constant to a morph-specific MGparameter. \\
	\hline
	Statistical 
	& \hyperlink{GcompVar}{Input variance adjuster} & 
	Added variance adjustment factor for generalized size comp. \\
	\\						
	& Deviation vectors & 
	Variance of deviation vectors is now specified with 2 parameters for standard error and auto-correlation (rho), so can be estimated.\\
	\\						
	& \hyperlink{Dirichlet}{Dirichlet multinomial} & 
	Dirichlet multinomial now a fleet-specific option; takes one parameter per fleet. \\
	\hline
	Parameters 
	& \hyperlink{paraOrder}{Parameter order} & The prior standard deviation column for all parameter lines has been moved before the prior type column.  This modification improves formatting output between integer and decimal inputs.\\ 
	\\
	& Density dependence & 
	Beginning of year summary biomass and the rec\_dev parameters are mapped to the "environmental" matrix so that parameters can be density-dependent.\\
	\\						
	& \hyperlink{tvOrder}{Re-order} & 
	Pay attention to the new order of the time-varying adjustments to parameters (block/trend, then environmental, then deviations). \\
	\\						
	& \hyperlink{time-vary}{Time-varying parameters} & 
	Long parameter lines for spawner-recruit relationship (SRR), catchability (Q), and tag parameters and complete re-vamp of the way that time-varying parameters are implemented for SRR and Q.  Now shares same internal code as mortality-growth and selectivity parameters for time-varying capabilities.\\
	\hline	
	\end{longtable}
\end{center}

%\pagebreak
%\begin{center}
%	\Large{In Process and Wishlist Items\\
%		for Future Versions}\\
%\end{center}

%\begin{center}
%	\begin{longtable}{p{2cm} p{3cm} p{10cm}}
%		Category & Item & Description\\
%		\hline
%		\endfirsthead
%		
%		Category & Item & Description\\
%		\hline
%		\endhead
%		
%		\hline
%		\endfoot
%		
%		\endlastfoot
		
%		In process & 
%		Bycatch Fleets & Bycatch fleet implementation \\
%		\\
%		&	Environmental survey of a deviation vector & Environmental survey data can be related to f(a deviation vector), but needs more Q\_link functions.\\
%		\\
%		& Retrospective & Make blocks work when doing retrospective analyses. \\
		
%		\hline
%		Wishlist & &
%		Tag-recapture major revamp\\\\
%		& & Area specific spawner-recruitment \\\\
%		& & Add automatic set-up for size selectivity option 6 and age selectivity option 17 given the data (comparable to the current capacity for cubic spline selectivity).\\\\
%		& & More error checking on read of empirical weight-at-age and composition data. \\\\
%		& & Brief output for data-limited applications will omit many of the detailed tables.\\\\
%		& & New version of age-specific K needed because current version is inefficient.\\\\
%		& & SIS output\\\\
%		& & Mean size in plus group uses a fixed erosion factor of 0.2; should be context specific. \\\\
%		& & Consider Tim Miller's state space model approach. \\\\
%		& & Add other options to CV growth patter for log(SD). \\\\
%		& & Add measure of auto-correlation in composition and in survey residuals.\\
%		\pagebreak
%		Wishlist & & Add SD report for survey expected values. \\\\
%	    & & Add calculation of Francis composition weighting method. \\\\
%		& & Year-specific MSY.\\	  	  				
%		\hline
%	\end{longtable}
%\end{center}
	
\end{document}



