
\section{Introduction}\label{sec:intro}

Fish population (aka "stock")  assessment models determine the impact of past fishing on the historical and current abundance of the population, evaluate sustainable rates of removals (catch), and project future levels of catch reflecting one or more risk-averse catch rules.  These catch rules are codified in regional Fishery Management Plans according to requirements of the Sustainable Fisheries Act.  In the U.S., approximately 500 federally managed fish and shellfish populations are managed under approximately 50 Fishery Management Plans.  About 200 of these populations are assessed each year, based on a prioritized schedule for their current status. Despite this, many minor species have never been quantitatively assessed.  Although the pace is slower than that for weather forecasting, fish stock assessments are operational models for fisheries management.

Assessment models typically assimilate annual catches, data on fish abundance from diverse surveys and fishery sources, and biological information regarding fish body size and proportions at age.  A suite of models is available depending on the degree of data availability and unique characteristics of the fish population or its fishery.  Where feasible, environmental time series are used as indicators of changes in population or observation processes, especially to improve the accuracy of the projections of abundance and sustainable catch into the future.  Such linkages are based principally on correlations given the challenge of conducting field observations on an appropriate scale.  The frontier of model development is in the rapid estimation of parameters to include random temporal effects, in the simultaneous modeling of a suite of interacting species, and in the explicit treatment of the spatial distribution of the population.

Assessment models are loosely coupled to other models. For example, an ocean-temperature or circulation model or benthic-habitat map may be directly included in the pre-processing of the fish abundance survey.  A time series of a derived ocean factor, like the North Atlantic Oscillation, can be included as an indicator of a change in a population process.  Output of a multi-decadal time series of derived fish abundance can be an input to ecosystem and economic models to better understand cumulative impacts and benefits. 

Stock Synthesis is an age- and size-structured assessment model in the class of models termed integrated analysis models. Stock Synthesis has evolved since its initial inception in order to model a wide range of fish populations and dynamics.  The most recent major revision to Stock Synthesis occurred in 2016, when version 3.30 was introduced. This new version of Stock Synthesis required major revisions to the input files relative to earlier versions (see the \hypertarget{ConvIssues}{Converting Files} section for more information). The acronym for Stock Synthesis has evolved over time with earlier versions being referred to as SS2 (SS v.2.+), SS3 (SS v.3.+), or just simply SS.  The current official abbreviation of Stock Synthesis is SS3 for all versions greater than SS v.3.+. 


SS3 has a population sub-model that simulates a stock's growth, maturity, fecundity, recruitment, movement, and mortality processes, an observation sub-model estimates expected values for various types of data, a statistical sub-model characterizes the data’s goodness of fit and obtains best-fitting parameters with associated variance, and a forecast sub-model projects needed management quantities.  SS3 outputs the quantities, with confidence intervals, needed to implement risk-averse fishery control rules. The model is coded in C++ with parameter estimation enabled by automatic differentiation (\href{http://www.admb-project.org}{admb}).  Windows, Linux, and iOS versions are available.  Output processing and associated tools are in R, and a graphical interface is in QT.  SS is available from NOAA’s VLAB. The rich feature set in SS allows it to be configured for a wide range of situations.  SS3 has become the basis for a large fraction of U.S. assessments and many other assessments around the world.  

This manual provides a guide for using SS. The guide contains a description of the input and output files and usage instructions. An overview and technical description of the model itself is in \citet{methotstock2013}. However, SS3 has continued to evolve and grow since the publication in 2013, with this manual reflecting the most up to date information regarding SS.  The model and a graphical user interface are available from the NOAA \href{https://vlab.noaa.gov/group/stock-synthesis/home}{VLAB}. The VLAB site also provides a user forum for posting questions and for accessing various additional materials.  An output processor package, r4ss, in R is available for download from CRAN or \href{https://github.com/r4ss/r4ss}{GitHub}.

Additional guidance for new users can be found in the SS3 \href{https://vlab.noaa.gov/group/stock-synthesis/document-library}{document library} within the NOAA VLAB website.  The "Begin Here - Introduction to Stock Synthesis" folder located in the Document Library contains step-by-step guidance for running Stock Synthesis.  

\pagebreak
		
\section{File Organization}\label{FileOrganization}		
	\subsection{Input Files}
	\begin{enumerate}
		\item starter.ss: required file containing filenames of the data file and the control file plus other run controls (required).
		\item datafile: file containing model dimensions and the data (required)
		\item control file: file containing set-up for the parameters (required)
		\item forecast.ss: file containing specifications for reference points and forecasts (required) 
		\item ss.par: previously created parameter file that can be read to overwrite the initial parameter values in the control file (optional)
		\item wtatage.ss: file containing empirical input of body weight by fleet and population and empirical fecundity-at-age (optional)
		\item runnumber.ss: file containing a single number used as run number in output to CumReport.sso and in the processing of profilevalues.ss (optional)
		\item profilevalues.ss: file contain special conditions for batch file processing (optional)
	\end{enumerate}
	
	\subsection{Output Files}
	\begin{enumerate}
		\item data.ss\textunderscore new: Contains a user-specified number of datafiles, generated through a parametric bootstrap procedure, and written sequentially to this file.
		\item control.ss\textunderscore new: Updated version of the control file with final parameter values replacing the initial parameter values.
		\item starter.ss\textunderscore new: New version of the starter file with annotations.
		\item Forecast.ss\textunderscore new: New version of the forecast file with annotations.
		\item warning.sso: This file contains a list of warnings generated during program execution.
		\item echoinput.sso: This file is produced while reading the input files and includes an annotated echo of the input. The sole purpose of this output file is debugging input errors.
		\item Report.sso: This file is the primary report file.
		\item ss\_summary.sso: Output file that contains all the likelihood components, parameters, derived quantities, total biomass, summary biomass, and catch. This file offers an abridged version of the report file that is useful for quick model evaluation. This file is only available in SS3 v.3.30.08.03 and greater.
		\item CompReport.sso: Observed and expected composition data in a list-based format.
		\item Forecast-report.sso: Output of management quantities and for forecasts.
		\item CumReport.sso: This file contains a brief version of the run output, output is appended to current content of file so results of several runs can be collected together.  This is useful when a batch of runs is being processed.
		\item Covar.sso: This file replaces the standard ADMB ss.cor with an output of the parameter and derived quantity correlations in database format.
		\item ss.par: This file contains all estimated and fixed parameters from the model run. 
		\item ss.std, ss.rep, ss.cor etc.:  Standard ADMB output files.
		\item checkup.sso: Contains details of selectivity parameters and resulting vectors.  This is written during the first call of the objective function.
		\item Gradient.dat: New for SS3 v.3.30, this file shows parameter gradients at the end of the run.
		\item rebuild.dat: Output formatted for direct input to Andre Punt's rebuilding analysis package.  Cumulative output is output to REBUILD.SS (useful when doing MCMC or profiles).
		\item SIS\_table.sso: Output formatted for reading into the NMFS Species Information System.
		\item Parmtrace.sso: Parameter values at each iteration.
		\item posteriors.sso, derived\_posteriors.sso, posterior\_vectors.sso: Files associated with MCMC.
	\end{enumerate}

	
	\subsection{Auxiliary Files}
	These files are additional files (e.g. excel files) which allow for exploration or understanding of specific parameterization which can assist in selecting appropriate starting values.  These files are available for download from the Vlab website. 
	\begin{enumerate}
		\item SS3-OUTPUT.xls: Excel file with macros to read report.sso and display results.
		\item SS\_330\_Control\_Setup.xls:
		\begin{enumerate}
			\item Shows how to setup an example control file for SS.
		\end{enumerate}
		\item SS\_330\_Data\_Input.xls:
		\begin{enumerate}
			\item Shows how to setup an example data input file for SS.
		\end{enumerate}
		\item SS\_330\_Starter\&Forecast.xls:
		\begin{enumerate}
			\item Shows how to setup an example starter and forecast input files for SS.
		\end{enumerate}

	\end{enumerate}

\pagebreak
		
\section{Starting Stock Synthesis}
SS3 is typically run through the command line interface, although it can also be called from another program, R, the SS-GUI or a script file (such as a DOS batch file). SS3 is compiled for Windows, Mac, and Linux operating systems. The memory requirements depend on the complexity of the model you run, but in general, SS3 will run much slower on computers with inadequate memory. See the Running Stock Synthesis section on page \pageref{sec:RunningSS} for additional notes on methods of running SS3.

Communication with the program is through text files.  When the program first starts, it reads the file starter.ss, which typically must be located in the same directory from which SS3 is being run.  The file starter.ss contains required input information plus references to other required input files, as described in the File Organization section on page \pageref{FileOrganization}.  The names of the control and data files must match the names specified in the starter.ss file.  File names, including starter.ss, are case-sensitive on Linux and Mac systems but not on Windows. The echoinput.sso file outputs how the executable read each input file and can be used for troubleshooting when trying to setup a model correctly.  Output from SS3 consists of text files containing specific keywords.  Output processing programs, such as the SS3 GUI, Excel, or R can search for these keywords and parse the specific information located below that keyword in the text file.

\pagebreak